\documentclass{article}
\usepackage[utf8]{inputenc}
\usepackage[serif]{pkige}
\usepackage{hyperref}
\usepackage{amsmath} 
\usepackage{amsfonts}
\usepackage{amssymb} 
\usepackage{url}



\newcommand{\vecm}{\operatorname{vec}}
\newcommand{\diagm}{\operatorname{diagm}}
\newcommand{\dotstar}{\mathbin{.*}}

\title{18.S096 PSET 1}
\author{IAP 2023}
\date{Due 1/25/2023}

\begin{document}

\maketitle

\subsection*{Problem 0 (4+4+4+4 points)}

The hyperbolic Corgi notebook may be found at 
\url{https://mit-c25.netlify.app/notebooks/1_hyperbolic_corgi}. 
Compute the $2 \times 2$ Jacobian matrix for each of the following image
transformations from that notebook:

\begin{enumerate}[label=(\alph*)]

\item rotate($\theta$):
$(x,y)\rightarrow 
(\cos(\theta)x + \sin(\theta)y, -\sin(\theta)x + \cos(\theta)y)$

\item hyperbolic\_rotate($\theta$): $(x,y)\rightarrow(\cosh(\theta)x+\sinh(\theta)y,\sinh(\theta)x + \cosh(\theta)y )$

\item nonlin\_shear($\theta$):
$(x, y) \rightarrow (x, y + \theta x^2)$

\item warp($\theta$):
$ (x, y) \rightarrow  \mbox{rotate}(\theta \sqrt{x^2+y^2})(x, y)$

\end{enumerate}


\subsection*{Problem 1 (5+4 points)}


\begin{enumerate}[label=(\alph*)]

\item Suppose that $L[x]$ is a linear operation (for $x$ in some vector space $V$, with outputs $L[x]$ in some other vector space $W$).   If $f(x) = L[x] + y$ for a constant $y \in W$, what is $f'(x)$ (in terms of $L$ and/or $y$)?

\item Give the derivatives of $f(A) = A^T$ (transpose) and $g(A) = 1 + \tr A$ (trace) as special cases of the rule you derived in the previous part. 

\end{enumerate}

\subsection*{Problem 2 (5+6+5+5 points)}


Calculate derivatives of each of the following functions in the requested forms---as a linear operator $f'(x)[dx]$, a Jacobian matrix, or a gradient $\nabla f$ ---as specified in each part.

\begin{enumerate}[label=(\alph*)]

\item $f(x) = x^T (A + \diagm(x))^2 x$, where the inputs $x \in \mathbb{R}^n$ are vectors, the outputs are scalars, $A = A^T$ is a constant \emph{symmetric} $n\times n$ matrix $\in \mathbb{R}^{n\times n}$, and $\diagm(x)$ denotes the  $n\times n$  diagonal matrix $\begin{pmatrix} x_1 & & \\ & x_2 & \\ & & \ddots \end{pmatrix}$.  Give the \textbf{gradient} $\nabla f$, such that $f'(x)dx = (\nabla f)^T dx$.

\item $f(x) = (A + yx^T)^{-1} b$, where the inputs $x$ and outputs $f(x)$ are $n$-component (column) vectors in $\mathbb{R}^n$, $y$ and $b$ are constant vectors $\in \mathbb{R}^n$, and $A$ is a constant $n\times n$ matrix $\in \mathbb{R}^{n\times n}$.   

\begin{enumerate}[label=(\roman*)]
\item Give $f'(x)$ as a \textbf{Jacobian} matrix.
\item If you are given $A^{-1}$, then you can compute $(A + yx^T)^{-1}$ and hence $f(x)$ for any $x$ in $\sim n^2$ scalar-arithmetic operations (i.e., roughly proportional to $n^2$, or in computer-science terms $\Theta(n^2)$ ``complexity''), using the ``Sherman--Morrison'' formula (Google it).  \textbf{Explain} how your Jacobian matrix can therefore also be computed in $\sim n^2$ operations for any $x$ given $A^{-1}$ (i.e. give a sequence of computational steps, each of which costs no more than $\sim n^2$ arithmetic).

\end{enumerate}

\item $f(x) = \frac{xx^T}{x^T x}$, with vector inputs $x \in \mathbb{R}^n$ and matrix outputs $f\in \mathbb{R}^{n\times n}$.   Give $f'(x)$ as a linear operator, i.e.~a linear formula for $f'(x)[dx]$.

\item $g(x) = \frac{xx^T}{x^T x} b$, with vector inputs $x \in \mathbb{R}^n$ and vector outputs $f\in \mathbb{R}^n$, where $b \in \mathbb{R}^n$ is a constant vector.   Give $g'(x)$ as a \textbf{Jacobian} matrix.

\end{enumerate}

\subsection*{Problem 3 (5+5+5 points)}


\begin{enumerate}[label=(\alph*)]

\item
Argue briefly that linear functions that map $n \times n$ matrices to $n \times n$ matrices themselves form
a vector space $V$.  What is the dimension of this vector space?

\item
Argue briefly that linear functions of 
$n \times n$ matrices of the form $ X \rightarrow AX$, where $A$ is $n \times n$,  form a vector space.
What is the dimension of this vector space?

\item
Argue briefly why it follows that there must be infinitely many linear functions $\in V$
that are not of the form $X \rightarrow AX$.

\end{enumerate}

\end{document}